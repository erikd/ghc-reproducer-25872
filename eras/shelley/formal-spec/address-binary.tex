\section{Binary Address Format}
\label{sec:address-binary}

\newcommand{\binary}[1]{\ensuremath{\mathsf{#1}}}

Excepting bootstrap addresses,
the binary format for every address has a 1-byte header, followed by a payload
(bootstrap addresses use the binary encoding from the Byron era).
The header is composed of two nibbles (two 4-bit segments),
indicating the address type and the network id.

$$
\begin{array}{r@{~=~}l}
    \mathsf{address} & \mathsf{header\_byte}|\mathsf{payload} \\
                     & \mathsf{addr\_type\_nibble}|\mathsf{network\_nibble}|\mathsf{payload}
\end{array}
$$

\subsection{Header, first nibble}
\label{sec:address-binary-header-first-nibble}

The first nibble of the header specifies the type of the address.
Bootstrap addresses can also be identified by the first nibble.

\begin{center}
\begin{tabular}{ |c|c|c|c| }
 \hline
 address type & payment credential & stake credential & header, first nibble \\
 \hline
 \hline
 base address & keyhash      & keyhash    & \binary{0000} \\
 base address & scripthash   & keyhash    & \binary{0001} \\
 base address & keyhash      & scripthash & \binary{0010} \\
 base address & scripthash   & scripthash & \binary{0011} \\
 \hline
 pointer address & keyhash    & ptr & \binary{0100} \\
 pointer address & scripthash & ptr & \binary{0101} \\
 \hline
 enterprise address & keyhash    & - & \binary{0110} \\
 enterprise address & scripthash & - & \binary{0111} \\
 \hline
 bootstrap address & keyhash    & - & \binary{1000} \\
 \hline
 stake address & - & keyhash & \binary{1110} \\
 stake address & - & scripthash & \binary{1111} \\
 \hline
 future formats & ? & ? & \binary{1001}-\binary{1101} \\
 \hline
\end{tabular}
\end{center}

\subsection{Header, second nibble}
\label{sec:address-binary-header-second-nibble}

Excepting bootstrap addresses, the second nibble of the header specifies the network.

\begin{center}
\begin{tabular}{ |c|c| }
 \hline
 network & header, second nibble \\
 \hline
 \hline
 testnets & \binary{0000} \\
 mainnet & \binary{0001} \\
 future networks & \binary{0010}-\binary{1111}\\
 \hline
\end{tabular}
\end{center}

\subsection{Header, examples}
\label{sec:address-binary-header-examples}

On a \textbf{testnet},
the header for a \textbf{pointer} address whose payment credential is a \textbf{keyhash} is:
$$\binary{00000100}$$

On \textbf{mainnet}, the header for a \textbf{pointer} address whose payment credential is a \textbf{keyhash} is:
$$\binary{00010100}$$

On \textbf{mainnet}, the header for a \textbf{pointer} address whose payment credential is a \textbf{scripthash} is:
$$\binary{00010101}$$

\subsection{Payloads}
\label{sec:address-binary-payloads}

The payload for the different address types are as follows:

\begin{center}
\begin{tabular}{ |c|c| }
 \hline
 address type & payload \\
 \hline
 \hline
 base address & two 28-byte bytestrings \\
 \hline
 pointer address & one 28-byte bytestring,
 and three variable-length unsigned integers \\
 \hline
 enterprise address & one 28-byte bytestrings \\
 \hline
 stake address & one 28-byte bytestrings \\
 \hline
\end{tabular}
\end{center}

The variable-length encoding used in pointers addresses is the base-128 representation
of the number, with the most significant bit of each byte indicating continuation
(this is sometimes called a variable-length quantity, or VLQ).
If the significant bit is $1$, then another byte follows.
